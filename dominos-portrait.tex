\documentclass{article}
\usepackage{amsmath,amssymb}
\usepackage{tikz}
\usetikzlibrary{math}
\usepackage[margin=1cm]{geometry}

\setlength{\parindent}{0cm}
\setlength{\lineskip}{10pt}

\newcommand{\hdomi}[2]{
    \begin{tikzpicture}
        \tikzmath{
            \dwidth = 8;
            \dheight = 3.5;
        }
    \draw[thick] (0,0) rectangle (\dwidth,\dheight);
    \draw (\dwidth/2,0) -- (\dwidth/2,\dheight);
    \node at (\dwidth/4,\dheight/2) {#1};
    \node at (3*\dwidth/4,\dheight/2) {#2};
    \end{tikzpicture}
}

\newcommand{\vdomi}[2]{
    \begin{tikzpicture}
        \tikzmath{
            \dwidth = 8;
            \dheight = 3.5;
        }
    \draw[thick] (0,0) rectangle (\dwidth,\dheight);
    \draw (0,\dheight/2) -- (\dwidth,\dheight/2);
    \node at (\dwidth/2,3*\dheight/4) {\Large #1};
    \node at (\dwidth/2,\dheight/4) {\Large #2};
    \end{tikzpicture}
}
\pagestyle{empty}
\begin{document}
\begin{center}
    \vdomi{START}{$5\leq x<8$}
\vdomi{$[5,8)$}{$x\geq 8$}
\vdomi{$[8,\infty)$}{$(2,\infty)\cup (-\infty,2)$}
\vdomi{$\mathbb{R}\setminus\{2\}$}{$-1<x<5$}
\vdomi{$(-1,5)$}{$[2,6]\setminus [-1,3]$}
\vdomi{$(3,6]$}{$[3,\infty)\setminus [5,\infty)$}
\vdomi{$[3,5)$}{$(-1,4]\cup (2,8)$}
\vdomi{$(-1,8)$}{$(-\infty,3)\cup [0,4)$}
\vdomi{$(-\infty, 4)$}{$[-4,3)\cap [0,4)$}
\vdomi{$[0,3)$}{$x\leq 3$}
\vdomi{$(-\infty,3]$}{$x>-1$}
\vdomi{$(-1,\infty)$}{$x<\pi$}
\vdomi{$(-\infty,\pi)$}{$2\leq x\leq 9$}
\vdomi{$[2,9]$}{$[2,3]\cap [-1,6]$}
\vdomi{$[2,3]$}{$(4,9)\cap [4,7]$}
\vdomi{$(4,7]$}{$(-\infty,3)\cup [0,\infty)$}
\vdomi{$\mathbb{R}$}{$-1<x\leq 5$}
\vdomi{$(-1,5]$}{$(-\infty,3)\setminus [0,4)$}
\vdomi{$(-\infty, 0)$}{$(-1,5)\setminus [-1,3)$}
\vdomi{$[3,5)$}{$[2,3]\cup (-1,2]$}
\vdomi{$(-1,3]$}{$[2,3]\cap (4,7]$}
\vdomi{$\emptyset$}{END}

\end{center}

\end{document}
